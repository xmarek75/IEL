\section{Příklad 1}
% Jako parametr zadejte skupinu (A-H)
\prvniZadani{E}
\maketitle
\noindent\makebox[\linewidth]{\rule{\textwidth}{0.3pt}}
Tento obvod budeme řešit pomocí metody postupného zjednodušování obvodu. Nejprve tedy zjednodušíme rezistory $R_3$ a $R_4$, které jsou zapojeny paralelně, zároveň můžeme sloučit napětí $U_1$ a $U_2$. Výsledný rezistor nazveme $R_{34}$ a výsledné napětí $U$, rovnou můžeme sloučit rezisor $R_{34}$ a $R_2$ a vyjde nám $R_{234}$ (obvod č.1).\\
$$U=U_1+U_2=115+55=170 V$$
$$R_{34}=\frac{R_3*R_4}{R_3+R_3}=\frac{100*340}{100+340}=77.2727\Omega$$
$$R_{234}=R_{2}+R_{34}=660+77.2727=737.2727\Omega$$\\

\begin{center}
\begin{circuitikz}
\draw
(0,-1) to[R, l=$R_1$, *-*] (3,0)
(3,0) to[SHORT, *-*] (6,0)
(3,0) to[R, l=$R_5$, *-*] (3,-2)
(6,0) to[R, l=$R_7$, -*] (6,-2)
(0,-1) to[R, l=$R_{234}$, *-*] (3,-2)
(3,-2) to[R,l=$R_6$, i=$I_{R6}$] (6,-2)
(3.5,-2.3) to[open,v=$U_{R6}$] (5.5,-2.3)
(-2,-1) to[SHORT, *-*] (0,-1)
(-2,-1) to[V,v=U] (-2,-4)
(-2,-4) to[SHORT, *-*] (0,-4)
(0,-4) to[R, l=$R_8$, -] (3,-4)
(3,-4) to[SHORT, *-*] (6,-4)
(6,-4)to[SHORT, *-*] (6,-2)
(0,-0.75) node{A}
(3,-2.25) node{C}
(3,0.25) node{B}
;
\end{circuitikz}
\end{center}
\\
\pagebreak
V této situaci lze využít možnosti tzv. transformace trojúhelník -> hvězda - při uzlu A bude rezistor $R_A$, při uzlu B bude $R_B$ a při uzlu C bude $R_C$. Odpory výsledných rezistorů vypočítáme a obvod bude vypadat následovně (obvod č.2):
\\
\\
\\
$$R_A=\frac{R_1*R_{234}}{R_1+R_{234}+R_5}=\frac{485*737.2727}{485+737.2727+575}=198.9555\Omega$$
$$R_B=\frac{R_1*R_5}{R_1+R_{234}+R_5}=\frac{485*575}{485+737.2727+575}=155.1657\Omega$$
$$R_C=\frac{R_{234}*R_5}{R_1+R_{234}+R_5}=\frac{737.2727*575}{485+737.2727+575}=235.8751\Omega$$
\\
\begin{center}
\begin{circuitikz}
\draw
(-1,-1) to[R, l=$R_A$, -*] (1,-1)
(1,-1) to[R, l=$R_B$, --] (3,0)
(1,-1) to[R, l=$R_C$, --] (3,-2)
(3,0) to[SHORT, *-*] (6,0)
(6,0) to[R, l=$R_7$, -*] (6,-2)
(3,-2) to[R,l=$R_6$, i=$I_{R6}$] (6,-2)
(3.5,-2.3) to[open,v=$U_{R6}$] (5.5,-2.3)
(-2,-1) to[short, --] (-1,-1)
(-2,-1) to[V,v=U] (-2,-4)
(-2,-4) to[SHORT, *-*] (0,-4)
(0,-4) to[R, l=$R_8$, -] (3,-4)
(3,-4) to[SHORT, *-*] (6,-4)
(6,-4)to[SHORT, *-*] (6,-2)
(-1,-0.75) node{A}
(3,-2.25) node{C}
(3,0.25) node{B}
;
\end{circuitikz}
\end{center}
Nyní máme rezistory $R_B$ a $R_7$ zapojeny sériově, tak je sloučíme do $R_{B7}$ a jejich odpory sečteme. Podobně sloučíme i rezistory $R_C$ a $R_6$ do $R_{C6}$.

$$R_{B7}=R_B+R_7=155.1657+255=410.1657\Omega$$
$$R_{C6}=R_C+R_6=235.8751+815=1050.8751\Omega$$
\\
Obvod bude vypadat následovně (obvod č.3):

\begin{center}
\begin{circuitikz}
\draw
(-1,-1) to[R, l=$R_A$, -*] (1,-1)
(1,-1) to[short, --](1,0)
(1,-1) to[short, --](1,-2)
(1,0) to[R, l=$R_{B7}$, --] (3,0)
(1,-2) to[R, l=$R_{C6}$, --] (3,-2)
(3,0) to[short, --](3,-1)
(3,-2) to[short, --](3,-1)
(3,-1) to[short, *-] (6,-1)
(-2,-1) to[short, --] (-1,-1)
(-2,-1) to[V,v=U] (-2,-4)
(-2,-4) to[SHORT, *-*] (0,-4)
(0,-4) to[R, l=$R_8$, -] (3,-4)
(3,-4) to[SHORT, *-*] (6,-4)
(6,-4)to[SHORT, *-*] (6,-1)

;
\end{circuitikz}
\end{center}
\\
 
\\
\pagebreak\\
Teď zjednodušíme rezistory $R_{B7}$ a $R_{C6}$, které jsou paralelně zapojeny a vznikne nám rezistor  $R_{B7C6}$ (obvod č.4).
\\
$$R_{B7C6}=\frac{R_{B7}*R_{C6}}{R_{B7}+R_{C6}} =\frac{410.1657*1050.8751}{410.1657+1050.8751}=295.0177\Omega$$  
\\
\begin{center}
\begin{circuitikz}
\draw
(-2,0) to[V,v=U] (-2,-3)
(-2,-3) to[R, l=$R_8$, -] (4,-3)
(-2,0) to[R, l=$R_A$,-] (1,0)
(1,0) to[R, l=$R_{B7C6}$, -] (4,0)
(4,-3)to[short, --](4,0)
;\\
\end{circuitikz}
\end{center}
\\
\\
Nyní už máme všechny rezistory zapojeny sériově, jejich odpory sečteme a vznikne nám $R_{EKV}$.
Vznikne nám tak obvod s jediným rezistorem $R_{EKV}$ a s napětím $U$. Díky tomu můžeme dopočítat proud 
(obvod č.5).
\\

$$R_{EKV}=R_A+R_{B7C6}+R_8=198.9555+295.0177+225=718.9732\Omega$$
$$I=\frac{U}{R_{EKV}}=\frac{170}{718.9732}=0.2364A$$
\begin{center}
\begin{circuitikz}
\draw
(-2,0) to[V,v=U] (-2,-3)
(-2,0) to[R, l=$R_{EKV}$, -] (3,0)
(-2,-3)to[short, --](3,-3)
(3,0)to[short, --](3,-3)
\end{circuitikz}
\end{center}
Teď začneme postupně vracet rezistory do obvodu a budeme počítat jejich hodnoty.
\\
\\
obvod č.4:
$$I_{R_{B7C6}}=I_{R_8}=I_{R_A}=I=0.2364A$$
$$U_{R_{B7C6}}={R_{B7C6}}*I=295.0177*0.2364=69.7422V$$
$$U_{R_A}=R_A*I=198.9555*0.2364=47.0331V$$
$$U_{R_8}=R_8*I=225*0.2364=53.19V$$\\
Kontrola správnosti výpočtů (I. Kirchoffův zákon):
$$U_{R_{B7C6}}+U_{R_8}+U_{R_A}-U=69.7422+47.0331+53.19-170=169.9653-170\approx0$$\\
\\
\pagebreak
\\
obvod č.3:
$$I_{R_{C6}}=\frac{U_{R_{B7C6}}}{R_{C6}}=\frac{69.7422}{1050.8751}=0.0664A$$
$$U_{R_C6}=I_{R_{C6}}*R_{C6}=0.0664*1050.8751=69.7781V$$
\\
obvod č.2:
$$I_{R_{6}}=I_{R_{C6}}=0.0664A$$
$$U_{R_{6}}=I_{R_{6}}*R_6=0.0664*815=54.116V$$
\\

\\
\begin{center}
\boldsymbol{$$I_{R_{6}}=0.0664A$$}\\
\boldsymbol{$$U_{R_{6}}=54.116V$$}
\end{center}