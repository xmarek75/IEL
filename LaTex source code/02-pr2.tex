\section{Příklad 2}
% Jako parametr zadejte skupinu (A-H)
\druhyZadani{B}
\noindent\makebox[\linewidth]{\rule{\textwidth}{0.3pt}}

Nejdříve si vytvoříme ekvivalentní obvod a to vzhledem k rezistoru $R_3$, u kterého chceme zjistit proud a napětí. Rovnou si vytvoříme i rovnici pro výpočet $I_{R_3}$.
$$I_{R_3}=\frac{U_i}{R_i+R_3}$$\\
\begin{center}
\begin{circuitikz}
\draw
(0,0) to[V,v=U] (0,-3)
(0,0) to[short, -,i=$I$] (1,0)
(0,0) to[R, l=$R_i$, -] (3,0)
(3,0) to[R, l=$R_3$, -] (3,-3)
(3,-3) to[short, -] (0,-3)
;
\end{circuitikz}
\end{center}
Teď musíme zjistit $R_i$ a $U_i$. \\
V původním obvodu odpojíme zdroj napětí a rezistor $R_3$.
\begin{center}
\begin{circuitikz}
\draw
(0,0) to[short,-] (0,-3)
(0,0) to[short, -*] (2,0)
(2,0) to[short, -] (2,1)
(2,0) to[short, -] (2,-1)
(2,1) to[R, l=$R_1$, -] (4,1)
(2,-1) to[R, l=$R_2$, -] (4,-1)
(4,1) to[short, *-o] (4,0.7)
(4,-1) to[short, *-o] (4,-0.7)
(4,1) to[R, l=$R_4$, -] (6,1)
(4,-1) to[R, l=$R_6$, -] (6,-1)
(6,1) to[R, l=$R_5$, -*] (6,-1)
(0,-3) to[short, -] (6,-3)
(6,-3) to[short, -] (6,-1)
;
\end{circuitikz}
\end{center}
\\
\pagebreak
Teď zjistíme $R_i$(=$R_{EKV}$) tak, že sloučíme všechny rezistory do jednoho.
\\
$$R_{45}=R_4+R_5=220+570=790\Omega$$\\
$$R_{12}=\frac{R_1*R_2}{R_1+R_2}=\frac{50*310}{50+310}=43.0556\Omega$$\\
$$R_{456}=\frac{R_{45}*R_6}{R_{45}+R_6}=\frac{790*200}{790+200}=159.596\Omega$$\\
$$R_i=R_{12}+R_{456}=43.0556+159.596=202.6516\Omega$$\\
$R_i$ už jsme zjistili, nyní je třeba zjistit $U_i$ - napěťové zdroje v původním obvodu zapojím zpátky, $R_3$ nechám odpojené a místo něj zde bude hledané napětí $U_i$.
\begin{center}
\begin{circuitikz}
\draw
(0,0) to[V,v=U] (0,-3)
(0,0) to[short, -*] (2,0)
(2,0) to[short, -] (2,1)
(2,0) to[short, -] (2,-1)
(2,1) to[R, l=$R_1$, -] (4,1)
(2,-1) to[R, l=$R_2$, -] (4,-1)
(4,1) to[short, *-o] (4,0.7)
(4,-1) to[short, *-o] (4,-0.7)
(4,1) to[R, l=$R_4$, -] (6,1)
(4,-1) to[R, l=$R_6$, -] (6,-1)
(6,1) to[R, l=$R_5$, -*] (6,-1)
(0,-3) to[short, -] (6,-3)
(6,-3) to[short, -] (6,-1)
;
\end{circuitikz}
\end{center}
Nyní musíme zjistit napětí na $R_1$ a na $R_2$.
$$U_{R_1}=\frac{U*R_1}{R_1+R_{45}}=\frac{100*50}{50+790}=5.9524V$$
$$U_{R_2}=\frac{U*R_2}{R_2+R_6}=\frac{100*310}{310+200}=60.7843V$$
$U_i$ vypočítáme jako rozdíl $U_{R_2}$ a $U_{R_1}$.
$$U_i=U_{R_2}-U_{R_1}=60.7843-5.9524=54.8319$$
Teď se vrátíme na začátek ke vzorci $I_{R_3}$ a dosadíme do něj hodnoty.
$$I_{R_3}=\frac{U_i}{R_i+R_3}=\frac{54.8319}{202.6516+610}=0.0675A$$
A teď už jen dopočítáme napětí $U_{R_3}$.
$$U_{R_3}=I_{R_3}*R_3=0.0675*610=41.175V$$

